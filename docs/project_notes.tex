\documentclass{article}

\usepackage{color}
\usepackage{todonotes}

\usepackage{datetime}
\newdate{last-update-date}{13}{12}{2016}
\date{\displaydate{last-update-date}}


\title{Notes on Genesis Implementation}
\author{Jos\'e Cambronero}

\begin{document}
\maketitle

\newcommand{\unclear}[1]{{\color{red} ??? \textbf{#1}}}

\begin{abstract}
A set of notes regarding the implementation of Fan Long's Genesis project
\end{abstract}

These are my personal notes as I become familiar with Genesis. Any statement made here is subject to change. Any errors here
are solely attributable to the author and in no way reflect upon the Genesis project.


\section{Source code notes}
We provide a simple overview of the source code, describing major packages/classes and general functionality. 


\subsection{genesis.node}
Building blocks for Genesis AST representation
\begin{itemize}
\item MyCtNode: Generic AST nodes used in Genesis, basically use reflection to create a single node type for every spoon node
 	\begin{itemize}
     		\item Each node has children
     		\item Nodes may also be a collection
     		\item Has generic helper methods like checking if two trees are equal, getting parent, getting children etc
	\end{itemize}	
\item MyNodeSig: generic container for "spoon type" for statement represented by a MyCtNode instance
\item MyNodeTrait:  handles additional leaf information for reflected tree, e.g. whether a leaf node is a UnaryOp etc
\end{itemize}

\subsection{genesis.schema}
Contains code used to generalize the before/after AST trees (low-level)
\begin{itemize}
	\item SchemaAbstractor: contains before/after trees (MyCtNode), along with context information (insides: MyNodeSig),
    and variable bindings. With this it can generalize a set of such information to a pre/post tree transformation (TransASTNode)
    It uses dfs to construct both pre/post trees. It generalizes context information (insides) to the common super signature.
	\begin{itemize}
		\item SchemaAbstractor.generalize returns a TransformSchema
	\end{itemize}
	
	\item TransformSchema: represents a single pre/post transformation with contextual information. Contains vistor pattern info
  for tree transformation. Also contains utilities to specify a TransformSchema by "hand".

	\item SchemaAdapter: API to apply a TransformSchema to an AST (see SchemaAdapter.applyTo), does some general checking related
  to types

	\item TransAST*: Defines AST representation of generalized AST used for transformations.
	\item TransASTVisitor: Abstract visitor class for the generalized AST
\end{itemize}

\subsection{genesis.transform}
Contains code for transformation when given source code representation. This seems to be a higher-level API for transformation, which delegates work to genesis.schema

\begin{itemize}
	\item ManualTransformRule: allows us to specify a transformation by hand
	\item CodeTransform: wrapper around a TransformSchema with some meta information about what the schema can generate
	\item CodeTransAbstractor: generalize the trees to TransformSchema instance and handles creation of generators for free generalized nodes
	\item CodeTransAdapter: higher level wrapper around SchemaAdapter to apply to an AST
\end{itemize}

\subsection{genesis.generator}
Holds different classes of generators for each type of node necessary. Each generator can estimate the cost
of generating a subtree

\begin{itemize}
	\item CopyGenerator: Copies a node from a tree and generates that as the subtree, along with replacement of variables if necessary
	\item EnumerateGenerator: seems to enumerate all possible subtrees for generation based on constraints
	\item RefBound: holds scope information for a reference
	\item ReferenceGenerator: handles the subclasses of CtReference, which have additional scope information linked to them
	\item TraitGenerator: generates nodes for trait
\end{itemize}


\subsection{genesis.analysis}
Set of analysis tools that seem mainly used in genesis.generator to correctly generate nodes and
for correctly determining if nodes can replace one another in genesis.schema
  
\begin{itemize} 
	\item StatementAnalyzer: collect statements in functions or in package
	\item StaticAnalyzer: collect some type info, check if a variable reference is valid, return fields/variable/methods in scope,
  check type info, etc
	\item TypeHelper: help inferring type info, check type compatibility/equality
\end{itemize}


\subsection{genesis.learning}
Code to learn transformations by generalizing over pairs of trees and using genetic programming to pick lower cost with high coverage

\begin{itemize}
	\item DecomposedCodePair: contains context possibilities for a before/after pair of trees
	\item TreeDiffer: narrows down the differences between two trees
	\item CodePairDecomposer: creates a list of DecomposedCodePairs, when given a pre and post AST. It first identifies the portion of the AST
	that actually corresponds to the patch changes. It then expands and finds includes some of the surrounding area, for generalization purposes. Given
	that it expands area etc it returns a list of these different pair options, rather than a single pair.
	\item Main: the actual learning, uses a genetic algorithm to pick generalizations that cover examples and minimize cost for generators
  used. Has various static classes for each round of learning. Each have some actual learning logic and then utilities for reading/writing
  intermediate results. A lot of the code complexity here seems to come from multithreading.
	\begin{itemize}
		\item Round1Task: Generalize seems to be called with pairs of DecomposedCodePairs (i.e. we generalize over 2 befores and 2 afters), keeps
  track of lowest cost and 2 indices to DecomposedCodePairs that created that lowest cost generalization.
  		\item Round2Task: seems to get the results from Round1Task, and extend generalization to include yet another DecomposedCodePair. Again,
      keeps lowest cost version.
      		\item The remainder of the code seems to be for the genetic programming algorithm (e.g. MutateTask/CrossOverTask)
	\end{itemize}
\end{itemize}

\subsection{genesis.space}
Code managing the search space for candidate patches

\begin{itemize}
	\item genesis.space.par: handwritten transformations from PAR paper
	\item SearchSpace: define the space of transformations that can be applied to generate a patch
	\begin{itemize}
		\item GenerationResult: keep track of patches generated by a particular transformation
		\item applyTo: apply a search space of transformations to a before MyCtNode, collect those candidates that typecheck
	\end{itemize}
\end{itemize}

\subsection{genesis.rewrite}
Code for rewriting original source code to modified source code
 
\begin{itemize} 
	\item CodeRewritePass: (interface) rewriting interface that takes single tree and rewrites
	\item CodePairRewritePass: (interface) rewriting interface that takes before and after tree and rewrites
	\item CodePairRewriteWrapperPass: wraps a CodeRewritePass to apply to before and after trees.
	\item LiteralRewritePass: rewrite before literal to after literal by adding 1 (if possible
	\item VarDeclEliminationPass: replace a variable with an expression (VarDeclReplacer) and remove declarations, used to remove variables taht are only used once. Allows more normalized code for generalization.
	\item CodeRewriter: take original source file, represent as tree, and rewrite tree by replacing original nodes with new nodes
	\item CodeStyleNormalizationPass:  normalizes code, for example creates empty body for loops missing a body, adds then/else blocks as needed
	\item CommutativityRewritePass: swap left and right operand for plus expressions
	\item RewritePassManager: coordinates multiple passes, as needed. Takes 2 trees, finds difference between them. Takes all passes and incrementally applies them to the before/after snippets from the previous iteration. Writes out the before and after code strings to disk.
\end{itemize}


\subsection{genesis.repair}
Applies rewrites, compiles, and validates that patched program correctly fixes test that exposed issue, and doesn't regress

\begin{itemize}
	\item genesis.repair.compiler: manages use of javac compiler (don't want to use build tools, as too slow otherwise)
	\item genesis.repair.localization: localizes possible source code lines for error
	\begin{itemize}
		\item StackTraceDefectLocalization: one instance of a defect localizer, uses stack trace to identify "suspicious" lines in the source
			\begin{itemize} 
				\item returns a list of suspicious locations, ordered by suspiciousness
			\end{itemize}
		\item SuspiciousLocation: encodes information about a possible source code location that is an issue, the suspiciousness score is
  currently set with a heuristic in StackTraceDefectLocalization
  	\end{itemize}
	\item genesis.repair.validation: validates that a patch fixes failed test case and doesn't regress
	\begin{itemize}
		\item Testcase, TestResult: contains information for a single test case and for the test result
		\item TestcaseExecutor: runs test case and collects necessary information
		\item TestingOracle: validate the patch, collect any failed cases
	\end{itemize}
\end{itemize}


\subsection{genesis.corpus}
Manages the corpus of collected patches for learning etc.

\begin{itemize}
	\item CodeCorpusDB: interacts with mysql database, to query genesis tables 
	\item CodeCorpusDir: manages the directory with the actual code downloaded from github. Gets pairs of revisions and parses into ASTs
	\item CorpusPatch, CorpusApp: classes to represent information for a patch/application
	\item CorpusUtils: wrapper around parsing into Genesis AST
\end{itemize}

\subsection{genesis.utils}
Some random utilities

\subsection{genesis.infrastructure}
General code for infrastructure (e.g. for building code etc)


\end{document}